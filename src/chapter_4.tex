\chapter{Energía}

La energía es la capacidad de un cuerpo para cambiar de estado físico.
Ejemplos de estados físicos son la posición, la velocidad, la temperatura, o la carga eléctrica de un sistema, entre otros.
Cada estado físico tiene asociado un tipo de energía diferente.

La transferencia de energía se llama trabajo y se denota como $W$.
Tanto la energía como el trabajo van a tener unidad de Joules $[J]$.
Por convención, si el trabajo es positivo significa que una fuerza externa le está entregando energía al sistema.

En el capítulo \ref{cha:dinamics} se vió que al aplicar una fuerza sobre un cuerpo, este puede cambiar su velocidad y posición.
En este capítulo se analizará cómo un sistema puede transferirle parte de su energía a otro mediante una fuerza, y qué magnitudes se conservan o no durante el proceso.


\section{Tipos de energía mecánica}

Un sistema puede acumular varios tipos de energía.
En mecánica clásica, se estudian los siguientes tipos de energía mecánica.

La energía cinética es aquella que tiene un cuerpo por estar en movimiento, ya que es proporcional a su velocidad.

\begin{mdframed}[style=DefinitionFrame]
    \begin{defn}
    \end{defn}
    \cusTi{Energía cinética}
    \begin{equation*}
        \sub{E}{cin} = \frac{m \, v^2}{2}
    \end{equation*}
\end{mdframed}

La energía potencial gravitatoria es aquella que tiene un cuerpo por estar a cierta distancia del centro de masa de un sistema.

\begin{mdframed}[style=DefinitionFrame]
    \begin{defn}
    \end{defn}
    \cusTi{Energía potencial gravitatoria}
    \begin{equation*}
        \sub{E}{gra} = m \, g \, h
    \end{equation*}
\end{mdframed}

La energía potencial elástica es aquella que tiene un cuerpo por estar comprimiendo un resorte.

\begin{mdframed}[style=DefinitionFrame]
    \begin{defn}
    \end{defn}
    \cusTi{Energía potencial elástica}
    \begin{equation*}
        \sub{E}{ela} = \frac{k \, x^2}{2}
    \end{equation*}
\end{mdframed}

La energía potencial es toda aquella que surge a partir de un campo de gradiente.

\begin{mdframed}[style=DefinitionFrame]
    \begin{defn}
    \end{defn}
    \cusTi{Energía potencial}
    \begin{equation*}
        \sub{E}{pot} = \sub{E}{gra} + \sub{E}{ela}
    \end{equation*}
\end{mdframed}

La energía mecánica total de un cuerpo es la suma de energía cinética y potencial que el cuerpo tenga.

\begin{mdframed}[style=DefinitionFrame]
    \begin{defn}
        \label{defn:mechanicEnergy}
    \end{defn}
    \cusTi{Energía mecánica}
    \begin{equation*}
        \sub{E}{mec} = \sub{E}{cin} + \sub{E}{pot}
    \end{equation*}
\end{mdframed}


\section{Trabajo}

El trabajo que hace una fuerza cuantifica la energía que, el agente que la realiza, le está transfiriendo a un sistema.
Los procesos que involucran cambios de energía mecánica se suelen caracterizar estudiando el desplazamiento (Def. \ref{defn:desplazamiento}) de un cuerpo o sistema.

Consideremos una trayectoria $\xyz(t)$ junto con un campo de fuerza $\vec{F}$ que afecta a todos los puntos del recorrido, tales como $\xyz_0$ y $\xyz_1$.

\begin{center}
    \def\svgwidth{0.8\linewidth}
    \input{./images/energia-1.pdf_tex}
\end{center}

Para el segmento recto comprendido entre los puntos $\xyz_0$ y $\xyz_1$, vemos que la componente tangente ($\sub{F}{tan}$) de la fuerza ($\vec{F}$) y el ángulo ($\theta$) que esta forma con el desplazamiento ($\Delta \xyz$) permanecen constantes.

\begin{center}
    \def\svgwidth{0.6\linewidth}
    \input{./images/energia-2.pdf_tex}
\end{center}

Por trigonometría vemos que:
\begin{equation*}
    \sub{F}{tan} = \nnorm{\vec{F}} \cos\inParentheses{\theta}
\end{equation*}

El trabajo se define multiplicando la componente de fuerza que está contibuyendo al sentido de movimiento por la distancia de desplazamiento.

\begin{mdframed}[style=DefinitionFrame]
    \begin{defn}
    \end{defn}
    \cusTi{Trabajo}
    \begin{equation*}
        W = \sub{F}{tan} \, \nnorm{\Delta \xyz}
    \end{equation*}
\end{mdframed}

Pero la situación anterior se puede poner engorrosa fácilmente.
Ya sea porque la fuerza varía en módulo o dirección para diferentes puntos del espacio, o porque la dirección de la trayectoria varía con el tiempo.
En cualquiera de estos casos, la definición anterior no valdría, y habría que estudiar el diferencial de trabajo ($\dif W$) de un campo vectorial de fuerza a lo largo de una trayectoria.

En el caso de la trayectoria anterior, se observa que fuera del tramo recto el ángulo $\theta$ varía y, en otra circunstancia, también podría hacerlo el largo de $\vec{F}$.

\begin{center}
    \def\svgwidth{0.8\linewidth}
    \input{./images/energia-3.pdf_tex}
\end{center}

Aplicando la propiedad $\vec{u} \cdot \vec{v} = \nnorm{\vec{u}} \, \nnorm{\vec{v}} \cos\inParentheses{\theta}$ para el producto interno y luego integrando, se tiene:
\begin{align*}
    \Delta W &= \nnorm{\vec{F}} \, \nnorm{\Delta \vec{r}} \cos\inParentheses{\theta}
    \\[1ex]
    &= \vec{F} \cdot \Delta \xyz
    \\[1ex]
    W = \int \dif W &= \int_C \vec{F} \cdot \dif \xyz
\end{align*}

\begin{mdframed}[style=DefinitionFrame]
    \begin{defn}
        \label{defn:workLineIntegral}
    \end{defn}
    \cusTi{Trabajo}
    \begin{equation*}
        W = \int_C \Vec{F} \cdot \dif \Vec{s}
    \end{equation*}
\end{mdframed}

\begin{itemize}
    \item
    Donde $\dif \Vec{s}$ (o $\dif \Vec{r}$ en algunas bibliografías) es el diferencial de desplazamiento, también llamado ``line element'' o simplemente diferencial del vector de posición.
    Está dado por un segmento de recta infinitesimal de la trayectoria $(C)$ y se define con un versor tangente $(\versor{t})$ a la curva:
    \[ \dif \Vec{s} = \versor{t} \, \dif s \]
    
    \item Donde $\dif s$ es el elemento de arco, que es el largo del diferencial de distancia:
    \[
      \dif s = \norm{\dfrac{\dif}{\dif t} \xyz(t)} \dif t  = \norm{\dif \Vec{s} \,}
    \]
\end{itemize}

La anterior es una integral que en Cálculo Multivariable se conoce como integral curvilínea de tipo 2.
Pero este método sería muy engorroso ya que requeriría conocer la expresión vectorial de los factores, para después calcular su producto interno.

No obstante, podemos valernos de que las fuerzas perpendiculares a la trayectoria no hacen trabajo.
Si se logra descomponer al campo de fuerzas en una componente tangencial y otra normal, sólo hace falta computar la integral para la componente tangencial para calcular el trabajo.
\begin{align*}
    W &= \int_C \Vec{F} \cdot \dif \Vec{s}
    \\
    &= \int_C \Vec{F} \cdot \versor{t} \, \dif s
    \\
    &= \int_C \inParentheses{\sub{F}{rad} \, \versor{r} + \sub{F}{tan} \, \versor{\theta}} \cdot \versor{t} \, \dif s
    \\
    &= \int_C \sub{F}{rad} \underbrace{\versor{r} \cdot \versor{t}}_0 \dif s
    + \int_C \sub{F}{tan} \, \versor{\theta} \cdot \versor{t} \, \dif s
    \\
    &= \int_C \underbrace{\norm{\sub{F}{tan} \, \versor{\theta}}}_{\sub{F}{tan}} \, \underbrace{\norm{\versor{t}}}_{=1} \, \underbrace{\cos{(\theta)}}_{=1} \dif s
\end{align*}

En tal caso, el producto interno del integrando pasa a ser una simple multiplicación entre magnitudes escalares.

\begin{mdframed}[style=PropertyFrame]
    \begin{prop}
    \end{prop}
    \cusTi{Trabajo de fuerzas tangentes}
    \begin{equation*}
        W = \int_C \sub{F}{tan} \, \dif s
    \end{equation*}
\end{mdframed}

Observar que $\norm{\sub{F}{tan}}$ puede no ser constante, mientras que $\theta=\ang{0}$ sí lo es.

En una circunferencia, donde el radio no varía, se puede hacer un cambio de variable para expresar el elemento de arco $(\dif s)$ como un diferencial de ángulo $(\dif \theta)$ y poder así computar la integral.

Coloquialmente hacemos tender los $\Delta$ a $\dif$ en la definición de longitud de arco (Sec. \ref{A:arcLength}).
Formalmente, estamos calculando el diferencial $\dif s$ como sigue.

El largo de arco en función del ángulo está dado por $s(\theta) = r \, \theta$ y su derivada es $s'(\theta) = r$ quedando el diferencial definido como $\dif s = r \, \dif \theta$.
Reemplazando $\dif s$ en la definición \ref{defn:workLineIntegral} se tiene:

\begin{mdframed}[style=PropertyFrame]
    \begin{prop}
        \label{prop:workOnCircumference}
    \end{prop}
    \cusTi{Trabajo de fuerzas tangentes a circunferencias}
    \begin{equation*}
        W = r \int_{\theta_0}^{\theta_1} \sub{F}{tan} \, \dif \theta
    \end{equation*}
\end{mdframed}


\subsection{Trabajo de la fuerza Peso}

Como se quiere calcular la integral de la fuerza Peso (Def.\ \ref{defn:weightForce}) que es constante, se aplica su definición teniendo en cuenta que $\norm{\Vec{g}} = g$, ya que está contenida en una sola dimensión.

\begin{center}
    \def\svgwidth{0.8\linewidth}
    \input{./images/energia-4.pdf_tex}
\end{center}

Para trayectorias rectas, como el arco de curva es un segmento de recta, se integra con respecto a la distancia:
\begin{align*}
    W &= \int_C F \cdot \dif s
    \\
    W &= m \, g \int_{y_0}^{y_1} \dif y
    \\
    W &= m \, g \, \Delta y = \Delta \sub{E}{gra}
\end{align*}

La fuerza Peso tiene en todo momento y lugar la misma dirección y sentido, porque es constante.
Si la trayectoria es circular, no se puede aplicar el cambio de variable polar (Prop. \ref{prop:workOnCircumference}) en la integral de línea a mansalva.
Como se mencionó, esto se puede hacer solo cuando la fuerza y el diferencial de distancia tienen igual dirección y sentido.
Por lo tanto, es necesario descomponer la fuerza peso para estudiar su componente tangencial:
\begin{align*}
    W &= \int_C F \cdot \dif s
    \\
    W &= r \int_{\theta_0}^{\theta_1} F \, \dif \theta
    \\
    W &= - m \, g \, r \int_{\theta_0}^{\theta_1} \sin(\theta)  \, \dif \theta
    \\
    W &= - m \, g \, r \, \big( \cos(\theta_1) - \cos(\theta_0) \big)
    = \sub{E}{gra}
\end{align*}


\subsection{Trabajo de la fuerza elástica}

Por comodidad de notación, se suele expresar el estiramiento $\Delta l$ como $x$ para definir la Fuerza Elástica (Def.\ \ref{defn:elasticForce}), ya que $l$ es la posición en el eje $x$ del extremo móvil del resorte.
Como la trayectoria es recta, el arco de curva es un segmento de recta, y se integra con respecto a la distancia:
\begin{align*}
    W &= \int_C F \cdot \dif s
    \\
    W &= - k \int_{x_0}^{x_1} x \, \dif x
    \\
    W &= - \frac{k \inParentheses{x_1^2 - x_0^2}}{2} = \Delta \sub{E}{ela}
\end{align*}


\section{Fuerzas conservativas y disipativas}

De manera intuitiva, uno puede esperar que los sistemas en la realidad siempre pierdan energía, y esto es cierto.
Por ejemplo, un reloj de péndulo eventualmente se va a detener por el rozamiento con el aire, a menos que tenga un mecanismo eléctrico que lo mantenga andando.
Un patinador que se lance entre dos rampas en su skate eventualmente va a frenarse, a menos que se impulse con las piernas en cada bajada.
Un resorte en movimiento oscilatorio, eventualmente va a frenar por la fricción con el fluido en el que se mueve.

Esto se debe a que los sistemas que se ven afectados por fuerzas disipativas, como la fricción, no conservan su energía mecánica.
Dicho de otra forma, un sistema conserva la energía mecánica solo si realiza trabajo mediante fuerzas conservativas.

Las fuerzas conservativas son producidas por campos de gradiente de fuerza.
Calcular el trabajo de un campo de gradiente en cualquier trayectoria del espacio verifica las siguientes implicaciones, que son equivalentes entre si:
\begin{itemize}
    \item \textbf{Independencia de la trayectoria:} El trabajo de un punto al otro es igual, independientemente del camino que se tome, por más que un camino sea más largo y requiera de más recorrido aplicando la fuerza.
    
    \item \textbf{Trabajo nulo en trayectorias cerradas:} El trabajo de todo camino que empiece y termine en el mismo punto es nulo, se tome el camino que se tome.
\end{itemize}

Las fuerzas conservativas son aquellas que cumplen las condiciones anteriores.
Por el contrario, las fuerzas disipativas son las que no cumplen alguna, y por ende las dos condiciones anteriores.


\section[Principio de conservación de la energía]{Principio de conservación \\ de la energía}

Un sistema puede intercambiar energía con su entorno en forma de trabajo, de calor, de masa y puede acumular energía en forma de energía cinética, potencial, interna, eléctrica, y otras.
La relación entre todas las formas de acumular energía de un sistema, y todas las formas de transferencia de energía, viene dada por el principio de conservación de la energía.
Este es la piedra angular de todas las formas de transferencias y energías mencionadas, aunque varias no se ven en este texto.

El principio establece que si hay variación de la energía acumulada por el sistema $(\sub{E}{sis})$, entonces también va a haber transferencia $(T)$ de energía con el medio ambiente:
\[ \Delta \sub{E}{sis} = \sum T \]

Como se ve en la ecuación anterior, si hay variación de energía hay transferencia con el medio.
Es decir, que si aumenta la energía del sistema disminuye la energía del medio y viceversa, de manera que la energía total, del medio más el sistema no varía, simplemente se transfiere de uno al otro:
\[ \Delta \sub{E}{tot} = 0 \]

En Mecánica Clásica se estudia un caso particular del principio de conservación de la energía, donde solo hay transferencia de energía por trabajo producido por una fuerza, y el sistema solo acumula energía mecánica (Def.\ \ref{defn:mechanicEnergy}).
Bajo estas condiciones, se tiene:
\begin{align*}
    \sum \sub{W}{no con} &= \Delta \sub{E}{mec}
    \\
    \sum \sub{W}{con} &= - \Delta \sub{E}{pot}
    \\
    \sum \sub{W}{tot} &= \Delta \sub{E}{cin}
\end{align*}

Nótese que la tercer ecuación se deduce de sumar las dos primeras.
Como la primer ecuación es la definición de trabajo de fuerzas disipativas, es razonable que la segunda reciba el signo negativo para poder realizar la suma.


\section{Momento lineal}

El momento o cantidad de movimiento es la razón por la cual en un choque entre dos cuerpos de masas muy distintas, el que sea muy masivo no va a alterar tanto su movimiento, mientras que el de menor masa va a cambiar drásticamente su velocidad.
Por ejemplo, un camión que choque a una moto casi no se va a inmutar, mientras que la moto probablemente salga arrojada, por la arbolada.

El momento o cantidad de movimiento de un cuerpo es una magnitud vectorial que se define como la masa del cuerpo por la velocidad del mismo.

\begin{mdframed}[style=DefinitionFrame]
    \begin{defn}
        \label{defn:momentum}
    \end{defn}
    \cusTi{Momento}
    \begin{equation*}
        \Vec{P} = m \, \Vec{v}
    \end{equation*}
\end{mdframed}


\section[Principio de conservación del momento]{Principio de conservación \\ del momento}

Este principio se deduce de la definición de fuerza (Def.\ \ref{defn:force}) del capítulo \ref{cha:dinamics}:
\[ \Vec{F} = m \, \frac{\Delta \Vec{v}}{\Delta t} \]

El principio de conservación del momento establece que la variación del momento lineal es igual a la fuerza externa neta que se aplique en un cuerpo.
Aplicando la definición de momento (Def.\ ~\ref{defn:momentum}) a la ecuación anterior, se tiene:
\[ \Vec{F} = \frac{\Delta \Vec{P}}{\Delta t} \]

Si definimos el impulso $(\Vec{I})$ como la variación del momento, el principio se puede expresar de la siguiente manera:
\[ \Vec{F} \, \Delta t = \underbrace{\Delta \Vec{P}}_{\Vec{I}} \]

Por lo tanto, si sobre un cuerpo no se aplican fuerzas externas, el momento lineal se conserva:

\begin{mdframed}[style=PropertyFrame]
    \begin{prop}
    \end{prop}
    \begin{equation*}
        \Delta \Vec{P} = 0 \iff \sum \sub{F}{ext} = 0
    \end{equation*}
\end{mdframed}


\section{Colisiones}

Como en los choques de cuerpos macroscópicos los intervalos de tiempos son muy cortos, sería poco preciso calcular una función de la fuerza con respecto al tiempo.
Calcular el trabajo (Def.\ \ref{defn:workLineIntegral}) sería más engorroso ya que la Fuerza no sería constante, por las deformaciones que sufren las estructuras de los cuerpos en los choques.

En estos casos se usa el principio de conservación del momento, ya que en todas las colisiones el momento se conserva.

Las colisiones se clasifican en varios tipos, que nos permiten identificar patrones en diferentes situaciones.


\subsection{Colisiones elásticas}

Las colisiones elásticas son una idealización de las colisiones reales.
En las colisiones elásticas, suponemos que la energía cinética se conserva:
\[ \Delta \sub{E}{cin} = 0 \]

Hay colisiones que son más aptas para ser estudiadas como elásticas.
Es decir que algunos casos de choques reales se pueden estudiar como colisiones elásticas comentiendo un error casi nulo.


\subsection{Colisiones inelásticas}

Durante un choque de dos cuerpos macroscópicos las estructuras de los cuerpos se deforman.
Cuando uno de los cuerpos es golpeado, parte de la energía que se le aplicó en forma de energía cinética se convierte en energía interna.
Es decir, sus moléculas empiezan a oscilar como resortes y estas vibraciones internas aumentan la temperatura del cuerpo.

El modelo de colisiónes inelásticas contempla la pérdida de energía cinética que se da durante el choque:
\[ \Delta \sub{E}{cin} \neq 0 \]


\subsection{Colisiones inelásticas plásticas}

Un caso particular de las colisiones inelásticas son las plásticas.
Al ser un caso particular de colisiones inelásticas, en las colisiones plásticas la energía no se conserva.
\[ \Delta E_{cin} \neq 0 \]

Normalmente en una colisión se espera que los cuerpos que chocan salgan disparados con diferentes velocidades.
Dos cuerpos $A$ y $B$ que colisionen plásticamente se quedan unidos después de chocar, moviéndose con la misma velocidad final.
\[ v_{FA} = v_{FB} = v_F \]


\subsection{Explosiones ``inelásticas''}

Si bien las explosiones no son colisiones, siguen el mismo modelo de análisis.
Las explosiones se pueden pensar como un choque plástico que ocurren al revés en el tiempo.

Como es imaginable, mucha energía cinética se disipa en calor:
\[ \Delta E_{cin} \neq 0 \]

De manera análoga a las colisiones plásticas, la velocidad inicial de los fragmentos es la misma:
\[ v_{0A} = v_{0B} = v_0 \]
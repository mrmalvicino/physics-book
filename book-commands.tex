% COMANDOS DE FORMATO
\newcommand{\sub}[2]{{{#1}_\textsl{{#2}}}} % #1 subíndice #2 (Usado cuando #2 es $texto$)
\newcommand{\scale}[2]{\text{\scalebox{#1}{$#2$}}} % Aplicar factor de escala #1 a la ecuación #2
\newcommand{\cusTi}[1]{\noindent\textbf{#1}} % Título de definiciones, propiedades y ejemplos
\newcommand{\cusTe}[1]{\vspace{2mm}\\\text{\hspace{\the\parindent}}#1} % Descripción de definiciones, propiedades y ejemplos
\newcommand{\noTi}[1]{\text{\hspace{\the\parindent}}#1} % Descripción de MyFrame1 que no tenga título
\newcommand{\concept}[1]{\vspace{1ex} \textsc{#1}} % Subtítulos sin jerarquía
\newcommand{\braces}[1]{{ \left\{ {#1} \right\} }} % #1 entre llaves
\newcommand{\sqb}[1]{{ \begin{bmatrix} #1 \end{bmatrix} }} % #1 entre corchetes
\newcommand{\bb}[1]{\left(#1\right)} % #1 entre paréntesis
\newcommand{\sfrac}[2]{#1/#2} % Fracciones #1/#2 para reemplazar el (muy lento) \usepackage{xfrac}
\newcommand{\captionSpace}{-0.8cm} % Espacio entre figuras y pie de foto

% COMANDOS PARA NOTACIÓN DE FUNCIONES
\newcommand{\barrow}[3]{\begin{bmatrix} \left. #1 \right|_{#2}^{#3} \end{bmatrix}} % Regla de Barrow de #1 para los extremos de integración #2 y #3
\newcommand{\fx}[2][f]{#1 \hspace{-0.5mm} \left( #2 \right)} % #1 en función de #2 con paréntesis
\newcommand{\ffx}[2][f]{#1 \hspace{-0.5mm} \begin{bmatrix} #2 \end{bmatrix}} % #1 en función de #2 con corchetes
\newcommand{\intProd}[2]{<\hspace{-0.8mm}#1,#2\hspace{-0.8mm}>} % Producto interno entre #1 y #2
\newcommand{\comb}[2]{\begin{pmatrix} {#1}\\{#2} \end{pmatrix}} % Combinatorio n=#1, m=#2
\newcommand{\media}[2]{\underset{#2}{\sub{#1}{med}}} % #1 media entre #2
\newcommand{\norm}[1]{{\left| {#1} \right|}} % Módulo de #1
\newcommand{\nnorm}[1]{{\left|\left| {#1} \right|\right|}} % Norma de #1
\newcommand{\trans}[1]{#1^*} % Matriz transpuesta de #1
\newcommand{\conj}[1]{\overline{#1}} % Conjugado de #1
\newcommand{\ave}[1]{\bar{#1}} % Valor promedio de #1
\newcommand{\rms}[1]{{\sub{#1}{ef}}} % Valor eficaz de #1
\newcommand{\peak}[1]{{\sub{#1}{pk}}} % Valor pico de #1

% COMANDOS PARA NOTACIÓN DE ELEMENTOS Y OPERADORES
\newcommand{\class}[1][1]{\mathcal{C}^{#1}} % Clase o cantidad de derivadas parciales contínuas
\newcommand{\versor}[1]{\hat{#1}} % Vector unitario #1
\newcommand{\fasor}[1]{\check{#1}} % Fasor #1
\newcommand{\iVer}{\versor{\imath}} % i versor
\newcommand{\jVer}{\versor{\jmath}} % j versor
\newcommand{\kVer}{\versor{k}} % k versor
\newcommand{\eVer}{\versor{\textbf{e}}} % Versor canónico
\newcommand{\tang}{\textbf{t}} % Vector tangente
\newcommand{\setO}{\varnothing} % Conjunto vacío
\newcommand{\setN}{\mathbb{N}} % Conjunto de los números naturales
\newcommand{\setZ}{\mathbb{Z}} % Conjunto de los números enteros
\newcommand{\setR}{\mathbb{R}} % Conjunto de los números reales
\newcommand{\setI}{\mathbb{I}} % Conjunto de los números imaginarios
\newcommand{\setC}{\mathbb{C}} % Conjunto de los números complejos
\newcommand{\iu}{\mathrm{i}\mkern1mu} % Unidad imaginaria o número i
\newcommand{\setV}{\mathbb{V}} % Espacio vectorial V
\newcommand{\setW}{\mathbb{W}} % Espacio vectorial W
\newcommand{\setK}{\mathbb{K}} % Cuerpo K
\newcommand{\ith}{i} % Valor i-ésimo para sumatorias y permutadores
\newcommand{\jth}{j} % Valor j-ésimo para sumatorias y permutadores
\newcommand{\kth}{k} % Valor k-ésimo para sumatorias y permutadores
\newcommand{\nth}{n} % Valor n-ésimo para sumatorias y permutadores
\newcommand{\Nth}{N} % Valor N-ésimo para sumatorias y permutadores
\newcommand{\mth}{m} % Valor m-ésimo para sumatorias y permutadores
\newcommand{\dif}{\textsl{d}} % Diferencial
\newcommand{\grad}{\Vec{\nabla}} % Gradiente a fin
\newcommand{\absurd}{\bot} % Absurdo o contradicción
\newcommand{\tq}{\hspace{1ex} \big/ \hspace{1ex}} % tal que
\DeclareMathOperator{\sgn}{sgn} % Función signo
\DeclareMathOperator{\artan}{artan} % Arco tangente
\DeclareMathOperator{\sinc}{sinc} % Seno de x sobre x
\DeclareMathOperator{\proy}{proy} % Proyección ortogonal
\DeclareMathOperator{\Nu}{Nu} % Núcleo
\DeclareMathOperator{\im}{Im} % Imagen
\DeclareMathOperator{\ran}{ran} % Rango
\DeclareMathOperator{\bi}{Bi} % Variable aleatoria binomial
\DeclareMathOperator{\be}{Be} % Variable aleatoria de Bernoulli
\DeclareMathOperator{\geo}{G} % Variable aleatoria geométrica
\DeclareMathOperator{\hip}{H} % Variable aleatoria hipergeométrica
\DeclareMathOperator{\po}{Po} % Variable aleatoria de Poisson
\DeclareMathOperator{\uni}{U} % Distribución uniforme
\DeclareMathOperator{\ex}{exp} % Distribución exponencial
\DeclareMathOperator{\nor}{N} % Distribución normal
\DeclareMathOperator{\ecm}{ECM} % Error cuadrático medio (MSE)

% COMANDOS PARA NOTACIÓN DE CONSTANTES Y MAGNITUDES
\newcommand{\weight}{\textsl{p}} % Peso
\newcommand{\xyz}{\vec{r}\hspace{0.05cm}} % Trayectoria [x(t),y(t),z(t)]
\newcommand{\cstcoulomb}{k_e} % Constante de Coulomb